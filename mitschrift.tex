\documentclass[11pt]{scrartcl}
\usepackage[utf8]{inputenc}
\usepackage[ngerman]{babel}
\usepackage[a4paper, left=2cm, right=2cm, top=2cm]{geometry}
\usepackage[hidelinks]{hyperref}
\usepackage{acro}
\usepackage{todonotes}
\usepackage{longtable}
\usepackage{tabu}


\acsetup{
	make-links ,
	index/use,
	trailing/activate = {dash}
}
%Acronyme definieren
\DeclareAcronym{PSM}{
	short = PSM,
	long = Pflanzenschutzmittel
}

\DeclareAcronym{VW}{
	short = VW,
	long = Volkswirtschaft,
}

\DeclareAcronym{BMEL}{
	short = BMEL,
	long = Bundesministerium für Ernährung und Landwirtschaft,
}

\DeclareAcronym{BWS}{
	short = BWS,
	long = Bruttowertschöpfung,
}

\DeclareAcronym{NWS}{
	short = NWS,
	long = Nettowertschöpfung,
}

\DeclareAcronym{EU}{
	short = EU,
	long = europäische Union,
}

\title{Mitschrift APO WiSe 21/22}
\subtitle{Eine grauenhafte und unverständliche Mitschrift}
\author{Tibor Weiß}
\begin{document}
\maketitle
\newpage
\tableofcontents
\newpage
\printacronyms[
	heading = section*,
	name = Abkürzungsverzeichnis,
	template = longtabu
]
\newpage
\section{Einfürhung in die Agrarpolitik}
Beginn der VL!

\subsection{Grunddaten zur deutschen Land\-wirt\-schaft und Wert\-schöpf\-ung}
%\todo{VL hat mit nächsten Punkt angefangen}
\subsubsection{Statistische Daten}
Um Grunddaten über die deutsche Landwirtschaft zu erhalten, kann man beim \ac{BMEL}, statistisches Bundesamt oder Statista nach Daten suchen.
Diese veröffentlichen in der Regel Primärquellen.
Daten vom Deutschen Bauernverband (Lobbygruppe) sollte man nur mit viel Vorsicht verwenden, bzw. die Primärquellen ausfindig machen.

\subsubsection{Wie wichtig ist die Landwirtschaft an der dt. \ac{VW}}
\Ac{BWS} war 2020 der deutschen Landwirtschaft war 20,2 Mrd Euro, die \ac{NWS} lag bei 15,9 Mrd Euro.
Die \ac{BWS} ist der Produktionswert abzgl. der Vorleistungen.
\ac{NWS} ist die \ac{BWS} abzl. Abschreibungen und Abgaben und zzgl. Ausgleichszahlungen.


\todo{Vorlesungmitschrift aus fremden Mitschriften vervollständigen, hier ist eine Fehlstelle}
\subsection{Landwirtschaftliche Strukturwandel und Determinanten}

Was zeichnet den aktuellen Strukturwandel in der Landwitschaft aus und warum findet dieser statt.
Ist dies als Vor- oder Nachteil zu werten?

\subsubsection{Was ist der Strukturwandel?}
\begin{itemize}
	\item{Änderung von Daten im Sektor}
	\begin{itemize}
		\item Betriebsgröße
		
		\item Betriebsstrukturen

		\item Produktivität der Faktoren (Arbeit, Kapital, Boden)
	\end{itemize}
	\item{Bewertung von Strukturwandel (\ac{VW}-Sicht)}

		Generell ist Strukturwandel erwünscht, da so eine produktivere Wirtschaft ermöglicht wird. Dies führt dazu, dass \glqq schwache\grqq{}  Betriebe ausscheiden, da diese nicht in der Lage sind, ein nachhaltiges Einkommen zu erzielen.
Landwirtschaft ist der Strukturwandel sehr langsam, aufgrund der langen Investitionszyklen
	\item{Soziale Härten durch Anpassungen}

Immer weniger Betriebe im Velraufe der zeit (beobachtung)

\end{itemize}


\subsubsection{Determinanten}

Druck auf die Produzenten
\begin{itemize}
	\item{Spezialisierung und Skaleneffekte}

		Skaleneffekte nutzen, durchschnittliche Stückkosten senken (Spezialisierung bzw. Vergrößerung der Betriebe) und dadurch einen wirtschaftlichen Vorteil ggü anderen Betrieben erhalten.
	\item{technischer Fortschritt (auch biologisch, chemisch und organisatorischer)}

		Mög\-lich\-keit der Pro\-duk\-tivi\-täts\-stei\-ger\-ung, zB. durch Mechanisierung, Verbesserung der Sorten (Zucht), wirksame \ac{PSM} oder bessere Organisierung der Arbeitsabläufe (genauere Wettervorhersagen, vereinfachte Kommunikation).
		Dadurch erhöhen sich die Produktionsmengen und die Preise fallen (Marktdiagramm).
		Durch eine gesteigerte Nachfrage, kann der Effekt der fallenden Preise ausgeglichen werden, bzw verringert werden.
		In der landwirtschaftlichen Pri\-mär\-pro\-duk\-tion ist eine lokale Steigerung der Nachfrage häu\-fig nur über Be\-völ\-ker\-ungs\-wachs\-tum zu realisieren.
	\item{Außerlandwirtschaftliche Beschäftigungsmöglichkeiten}

		Abwanderung von Arbeitskräften, Notwendigkeit der Produktivitätssteigerung des Produktionsfaktors Mensch.
	\item{ungesicherte Hofnachfolge}

		Familienbetriebe (Von Inhaber geführte Betriebe) geben bei fehlendem Nachfolger (eigenes Kind) in der Regel auf (Beim Verkauf werden die Flächen in der Regel von anderen Betrieben gekauft und nicht von neuen Betrieben gekauft).
		Bei nicht vom Inhaber geführten Betrieben (Genossenschaften, AGs\ldots) wird ein neuer Verwalter, Betriebsleiter, Geschäftsführer\ldots von den Eigentümer eingestellt.
	\item{internationaler Wettbewerb}

		Produkte werden auf dem Weltmarkt gehandelt und beeinflussen daher die lokalen Preise.
		In den meisten Fäl\-len ist in einer Region die Produktion gün\-sti\-ger, sodass der Weltmarktpreis unter dem lokalen Preisgleichgewicht liegt.
		Sollte der Weltmarktpreis den lokalen Preis (deutlich) anheben, werden sich die Produktionsfaktoren verteuern (idR das knappste, derzeit Boden), da selbst mit einem höherem Faktorpreis ein Gewinn erzielt werden kann. 

		Aktuell (Herbst 2021) liegen die Weltmarktpreise für Getriede deutlich höher als die Produktionskosten.
		Wenn der Binnenmarkt am Welthandel teilnimmt, wird der Preis des Welthandels diktiert.
		Es entsteht eine Konsumentenrente (idR) oder Produzentenrente (aktuell) und führt zu Importen bzw. Exporten.

	\item{gesetzliche Auflagen}

	\item{Gesellschaftliche Anforderungen}

		82 Millionen Agrar-Experten in DE, welche unter dem Dunning-Kruger Effekt leiden.
	\item{kritische öff. Diskussion über die Landwirtschaft}
\end{itemize}

\subsection{Ziele der \ac{EU} Agrarpolitik und der deutschen Agrarpolitik}
%Beginn VL am 26.10.21
Warum werden Ziele (in der Politik) ungenau formuliert?
\begin{itemize}
	\item Um nicht an den Zielen gemessen werden zu können (Scheitern/Benotung uä)

		Schröder hatte damals eine bestimmte Arbeitslosenquote versprochen, dies wurde von den Medien ausgenutzt.

		Künast hatte eine Quote von 20\% Anbaufläche von ökologischer Landwirtschaft als Ziel genannt.
		Mit 7\% wurde das Ziel deutlich verfehlt, dies konnte entsprechend instrumentalisiert werden.

	\item Ziele der agrarpolitischen Entscheidungsträger sind (wahrscheinlich) ungleich zu denen der Agrar- und Umweltpolitik

		Politiker müssen in der Regel ihr Wählerstimmen maximieren - Landwirte/landwirtschaftlich nahe Unternehmen sind ein relativ kleines Wählerpotential.
		Künast hatte das Landwirtschaftsministerium um Verbraucher erweitert, um das Wählerpotential der Grünen zu erweitern.

	\item Viele Umwelt- und Agrarpolitische Maßnahmen haben einen sehr langen Zeithorizont

		Eine Veränderung der Düngung von landwirtschaftlichen Flächen hat einen Effekt auf die Nitratproblematik, allerdings sind diese erst nach 10 Jahren zu erkennen.

	\item Produktivitätssteigerung und bessere Stellung der Landwirtschaft ggü anderen wirtschaftlichen Branchen
		\todo{Vorlesung am 26.10.21 verlassen}
\end{itemize}

Zielkonflikte in der landwirtschaftlichen Agrarpolitik entstehen regelmäßig.
Es gibt viele Ziele, welche in direkter Konkurrenz zueinander stehen, zB CO\textsubscript{2}-Minderung versus Tierwohl.
CO\textsubscript{2}-Minderungen über Stilllegung von Mooren steht im Konflikt mit der Einkommenspolitik in der Agrarpolitik. Dies könnte man zB über Entschädigungen entschärfen.

Zukünftig werden diese und weitere Zielkonflikte die Agrarpolitik dominieren.
In der breiten Bevölkerung (und tlw auch Politik) fehlt häufig das fachliche Verständniss um solche Abschätzungen bewerten zu können.

\subsection{Gründe für staatliche Eingriffe in den Agrarsektor}

\paragraph{Marktversagen}
\begin{itemize}
	\item Ressourcenverteilung
		Bei externen Effekten sollte der Staat eingreifen, um sicherzustellen, dass die Ressourcen möglichst effizient verteilt werden.

	\item Viele Umweltgüter haben an sich keinen Preis.
Aufgrund der Martwirtschaft werden diese Güter relativ stark nachgefragt werden.
Die Politik muss dann entsprechend Regeln über Preise (CO\textsubscript{2}-Preise) in den Markt eingreifen.

\item Informationsassymetrie
	
	siehe Vorlesung Produktqualität

\item Marktmacht

\item Bei einem technisch vernünftigen Marktergebnis können soziale Probleme entstehen.
	Landwirte haben systematische Beeinträchtigung über die Risiken des Wetters.
\end{itemize}

Marktversagen rechtfertigt einen Eingriff der Politik.


\section{Landwirtschaftliche Einkommens- und Sozialpolitik}
\todo{hier vermutlich Ende VL 26.10.21}
\subsection{Einkommensdisparität der Landwirtschaft}


\subsection{Landwirtschaftliche Sozialpolitik}
\end{document}
