\documentclass[11pt]{scrartcl}
\usepackage[utf8]{inputenc}
\usepackage[ngerman]{babel}
\usepackage[a4paper, left=2cm, right=2cm, top=2cm]{geometry}
\usepackage[hidelinks]{hyperref}
\usepackage{acro}
\usepackage{todonotes}
\usepackage{longtable}
\usepackage{tabu}


\acsetup{
	make-links ,
	index/use,
	trailing/activate = {dash}
}
%Acronyme definieren
\DeclareAcronym{ED}{
	short = ED,
	long = Einkommensdisparität,
}
\DeclareAcronym{LAK}{
	short = LAK,
	long = landwirtschaftliche Alterskasse,
}
\DeclareAcronym{WGP}{
	short = WGP,
	long = Wertgrenzproduktivität,
}
\DeclareAcronym{PSM}{
	short = PSM,
	long = Pflanzenschutzmittel
}
\DeclareAcronym{GAP}{
	short = GAP,
	long = gemeinsame Agrarpolitik,
}
\DeclareAcronym{AG}{
	short = AG,
	long = Arbeitgeber,
}
\DeclareAcronym{VW}{
	short = VW,
	long = Volkswirtschaft,
}

\DeclareAcronym{BMEL}{
	short = BMEL,
	long = Bundesministerium für Ernährung und Landwirtschaft,
}

\DeclareAcronym{BWS}{
	short = BWS,
	long = Bruttowertschöpfung,
}

\DeclareAcronym{NWS}{
	short = NWS,
	long = Nettowertschöpfung,
}

\DeclareAcronym{EU}{
	short = EU,
	long = europäische Union,
}

\title{Mitschrift APO WiSe 21/22}
\subtitle{Eine grauenhafte und unverständliche Mitschrift}
\author{Tibor Weiß}
\begin{document}
\maketitle
\newpage
\tableofcontents
\newpage
\printacronyms[
	heading = section*,
	name = Abkürzungsverzeichnis,
	template = longtabu
]
\newpage
\section{Einfürhung in die Agrarpolitik}
Beginn der VL!

\subsection{Grunddaten zur deutschen Land\-wirt\-schaft und Wert\-schöpf\-ung}
%\todo{VL hat mit nächsten Punkt angefangen}
\subsubsection{Statistische Daten}
Um Grunddaten über die deutsche Landwirtschaft zu erhalten, kann man beim \ac{BMEL}, statistisches Bundesamt oder Statista nach Daten suchen.
Diese veröffentlichen in der Regel Primärquellen.
Daten vom Deutschen Bauernverband (Lobbygruppe) sollte man nur mit viel Vorsicht verwenden, bzw. die Primärquellen ausfindig machen.

\subsubsection{Wie wichtig ist die Landwirtschaft an der dt. \ac{VW}}
\Ac{BWS} war 2020 der deutschen Landwirtschaft war 20,2 Mrd Euro, die \ac{NWS} lag bei 15,9 Mrd Euro.
Die \ac{BWS} ist der Produktionswert abzgl. der Vorleistungen.
\ac{NWS} ist die \ac{BWS} abzl. Abschreibungen und Abgaben und zzgl. Ausgleichszahlungen.


\todo{Vorlesungmitschrift aus fremden Mitschriften vervollständigen, hier ist eine Fehlstelle}
\subsection{Landwirtschaftliche Strukturwandel und Determinanten}

Was zeichnet den aktuellen Strukturwandel in der Landwitschaft aus und warum findet dieser statt.
Ist dies als Vor- oder Nachteil zu werten?

\subsubsection{Was ist der Strukturwandel?}
\begin{itemize}
	\item{Änderung von Daten im Sektor}
	\begin{itemize}
		\item Betriebsgröße
		
		\item Betriebsstrukturen

		\item Produktivität der Faktoren (Arbeit, Kapital, Boden)
	\end{itemize}
	\item{Bewertung von Strukturwandel (\ac{VW}-Sicht)}

		Generell ist Strukturwandel erwünscht, da so eine produktivere Wirtschaft ermöglicht wird. Dies führt dazu, dass \glqq schwache\grqq{}  Betriebe ausscheiden, da diese nicht in der Lage sind, ein nachhaltiges Einkommen zu erzielen.
Landwirtschaft ist der Strukturwandel sehr langsam, aufgrund der langen Investitionszyklen
	\item{Soziale Härten durch Anpassungen}

Immer weniger Betriebe im Velraufe der zeit (beobachtung)

\end{itemize}


\subsubsection{Determinanten}

Druck auf die Produzenten
\begin{itemize}
	\item{Spezialisierung und Skaleneffekte}

		Skaleneffekte nutzen, durchschnittliche Stückkosten senken (Spezialisierung bzw. Vergrößerung der Betriebe) und dadurch einen wirtschaftlichen Vorteil ggü anderen Betrieben erhalten.
	\item{technischer Fortschritt (auch biologisch, chemisch und organisatorischer)}

		Mög\-lich\-keit der Pro\-duk\-tivi\-täts\-stei\-ger\-ung, zB. durch Mechanisierung, Verbesserung der Sorten (Zucht), wirksame \ac{PSM} oder bessere Organisierung der Arbeitsabläufe (genauere Wettervorhersagen, vereinfachte Kommunikation).
		Dadurch erhöhen sich die Produktionsmengen und die Preise fallen (Marktdiagramm).
		Durch eine gesteigerte Nachfrage, kann der Effekt der fallenden Preise ausgeglichen werden, bzw verringert werden.
		In der landwirtschaftlichen Pri\-mär\-pro\-duk\-tion ist eine lokale Steigerung der Nachfrage häu\-fig nur über Be\-völ\-ker\-ungs\-wachs\-tum zu realisieren.
	\item{Außerlandwirtschaftliche Beschäftigungsmöglichkeiten}

		Abwanderung von Arbeitskräften, Notwendigkeit der Produktivitätssteigerung des Produktionsfaktors Mensch.
	\item{ungesicherte Hofnachfolge}

		Familienbetriebe (Von Inhaber geführte Betriebe) geben bei fehlendem Nachfolger (eigenes Kind) in der Regel auf (Beim Verkauf werden die Flächen in der Regel von anderen Betrieben gekauft und nicht von neuen Betrieben gekauft).
		Bei nicht vom Inhaber geführten Betrieben (Genossenschaften, AGs\ldots) wird ein neuer Verwalter, Betriebsleiter, Geschäftsführer\ldots von den Eigentümer eingestellt.
	\item{internationaler Wettbewerb}

		Produkte werden auf dem Weltmarkt gehandelt und beeinflussen daher die lokalen Preise.
		In den meisten Fäl\-len ist in einer Region die Produktion gün\-sti\-ger, sodass der Weltmarktpreis unter dem lokalen Preisgleichgewicht liegt.
		Sollte der Weltmarktpreis den lokalen Preis (deutlich) anheben, werden sich die Produktionsfaktoren verteuern (idR das knappste, derzeit Boden), da selbst mit einem höherem Faktorpreis ein Gewinn erzielt werden kann. 

		Aktuell (Herbst 2021) liegen die Weltmarktpreise für Getriede deutlich höher als die Produktionskosten.
		Wenn der Binnenmarkt am Welthandel teilnimmt, wird der Preis des Welthandels diktiert.
		Es entsteht eine Konsumentenrente (idR) oder Produzentenrente (aktuell) und führt zu Importen bzw. Exporten.

	\item{gesetzliche Auflagen}

	\item{Gesellschaftliche Anforderungen}

		82 Millionen Agrar-Experten in DE, welche unter dem Dunning-Kruger Effekt leiden.
	\item{kritische öff. Diskussion über die Landwirtschaft}
\end{itemize}

\subsection{Ziele der \ac{EU} Agrarpolitik und der deutschen Agrarpolitik}
%Beginn VL am 26.10.21
Warum werden Ziele (in der Politik) ungenau formuliert?
\begin{itemize}
	\item Um nicht an den Zielen gemessen werden zu können (Scheitern/Benotung uä)

		Schröder hatte damals eine bestimmte Arbeitslosenquote versprochen, dies wurde von den Medien ausgenutzt.

		Künast hatte eine Quote von 20\% Anbaufläche von ökologischer Landwirtschaft als Ziel genannt.
		Mit 7\% wurde das Ziel deutlich verfehlt, dies konnte entsprechend instrumentalisiert werden.

	\item Ziele der agrarpolitischen Entscheidungsträger sind (wahrscheinlich) ungleich zu denen der Agrar- und Umweltpolitik

		Politiker müssen in der Regel ihr Wählerstimmen maximieren - Landwirte/landwirtschaftlich nahe Unternehmen sind ein relativ kleines Wählerpotential.
		Künast hatte das Landwirtschaftsministerium um Verbraucher erweitert, um das Wählerpotential der Grünen zu erweitern.

	\item Viele Umwelt- und Agrarpolitische Maßnahmen haben einen sehr langen Zeithorizont

		Eine Veränderung der Düngung von landwirtschaftlichen Flächen hat einen Effekt auf die Nitratproblematik, allerdings sind diese erst nach 10 Jahren zu erkennen.

	\item Produktivitätssteigerung und bessere Stellung der Landwirtschaft ggü anderen wirtschaftlichen Branchen
		\todo{Vorlesung am 26.10.21 verlassen}
\end{itemize}

Zielkonflikte in der landwirtschaftlichen Agrarpolitik entstehen regelmäßig.
Es gibt viele Ziele, welche in direkter Konkurrenz zueinander stehen, zB CO\textsubscript{2}-Minderung versus Tierwohl.
CO\textsubscript{2}-Minderungen über Stilllegung von Mooren steht im Konflikt mit der Einkommenspolitik in der Agrarpolitik. Dies könnte man zB über Entschädigungen entschärfen.

Zukünftig werden diese und weitere Zielkonflikte die Agrarpolitik dominieren.
In der breiten Bevölkerung (und tlw auch Politik) fehlt häufig das fachliche Verständniss um solche Abschätzungen bewerten zu können.

\subsection{Gründe für staatliche Eingriffe in den Agrarsektor}

\paragraph{Marktversagen} \todo{Wiederholung VL am 26.10.21}
\begin{itemize}
	\item Ressourcenverteilung
		Bei externen Effekten sollte der Staat eingreifen, um sicherzustellen, dass die Ressourcen möglichst effizient verteilt werden.

	\item Viele Umweltgüter haben an sich keinen Preis.
Aufgrund der Martwirtschaft werden diese Güter relativ stark nachgefragt werden.
Die Politik muss dann entsprechend Regeln über Preise (CO\textsubscript{2}-Preise) in den Markt eingreifen.

\item Informationsassymetrie
	
	siehe Vorlesung Produktqualität

\item Marktmacht

\item Bei einem technisch vernünftigen Marktergebnis können soziale Probleme entstehen.
	Landwirte haben systematische Beeinträchtigung über die Risiken des Wetters.
\end{itemize}

Marktversagen rechtfertigt einen Eingriff der Politik.


\section{Landwirtschaftliche Einkommens- und Sozialpolitik}
\todo{hier Ende VL 26.10.21}
\subsection{Einkommensdisparität der Landwirtschaft}
\acl{ED}
\begin{itemize}
	\item innere
	\item äußere
\end{itemize}

\subsubsection{Testbetriebsnetz}
Das Testbetriebsnetz wird genutzt, um die \ac{ED} zu ermitteln.
Es wird nicht nur das Einkommen erfasst, sondern auch viele sekundäre Daten wie Finanzierung oder Ausstattung mit AK.
Die Veröffentlichung erfolgt im \glqq Agrarpolitischen Bericht der Bundesregierung\grqq{}, wobei die Ergebnisse auf die Grundgesamtheit in Deutschland hochgerechnet werden.

Für die Vergleichsrechnung werden nur Haupterwerbsbetriebe an.
\begin{itemize}
	\item Der Gewinn wird für die Entlohnung der Arbeit und die eingesetzten Produktionsfakoren.

	\item Der Betriebsinhaber wird der durchschnittliche Bruttolohn (ohne \ac{AG}-Anteil) angesetzt. Für mitarbeitende Familienangehörige inkl. \ac{AG}-Anteil.

	\item Betriebsleiter wird abhängig vom Umsatz einen höheren Lohn bekommen. Kalkulatorisch mit 7€ mehr Lohn je 1000€ mehr Umsatz

	\item Zinsansatz für das eingesetzte Eigenkapital

	\item Ziel ist einen \glqq durchschnittlichen Selbständigen\grqq{} zu repräsentieren.
\end{itemize}

Im Agrarpolitischen Bericht wird der Unterschied in \% angegeben um die äußere \ac{ED} darzustellen.
In den meisten Jahren, ist dieser Prozentsatz negativ, dies zeigt, dass viele Betriebe zu teuer wirtschaften.
Häufig stehen größere Betriebe deutlich besser da, als kleinere Betriebe.

\subsubsection{spezielle Probleme der Vergleichsrechnung} existieren bei vergleichen zwischen den Jahren, muss man berücksichtigen, dass sich das Testbetriebsnetz ständig verändert.
Land wird in der Regel zum Anschaffungswert und nicht zum Verkehrswert gerechnet.
Aufgrund der steigenden Bodenpreisen wird dies nicht berücksichtigt, somit ist der Zinsansatz deutlich unterschätzt.
Außerlandwirschaftliches Einkommen wird nicht betrachtet.
Des weiteren wird das Bruttoeinkommen und nicht das Nettoeinkommen verglichen.

\subsubsection{verfügbares Einkommen} ist für die Einschätzugn der sozialen Lage der Landwirtschaft ist eine Berechnung des verfügbaren Einkommens ein guter Ansatzpunkt.
Dabei werden alle Einkommen aller Familienmitglider (Haushalt) nach den Steuern und sozialen Abgaben berechnet.

Eine \ac{ED} zu anderen, vergleichbaren Haushalten ist (häufig) nicht zu erkennen.
Allerdings werden Altenteileraufwendungen und ähnliche Aufwendungen nicht berücksichtigt.
Des weiteren sind in landwirtschaftlichen Haushalten häufig größer (mehr Kinder).

\subsection{Landwirtschaftliche Sozialpolitik}

\subsubsection{Prinzipien der Sozialpolitik}

\begin{itemize}
	\item Äquivalenzprinzip - mehr Einzahlung, mehr Leistung

		hohe Bereitschaft der Versichungsnehmer ihre eigene wirtschaftliche Stellung zu verbessern
	\item Solidaritätsprinzip - mehr Einzahlung - gleiche Leistung, zB Krankenversicherung, Beitrag abhängig vom Lohn, trotzdem gleiche Leistung

		Eigenverantwortung und Leistungsbereitschaft werden nicht belohnt oä, daher gering und tendenz zur Rationalitätsfalle zur Ausweitung der Leistungsausbreitung und dadurch eine Beitragsleistungsausweitung

	\item Versorgungsprinzip - keine Beiträge, aber Leistung, zB Hartz IV

		Wird nur verwendet, wenn nicht anders möglich!
	\item Subsidaritätsprinzip - Selbsthilfe, Hilfe aus dem nahen Umfeld, zB Versicherungen mit Selbstbeteiligung
\end{itemize}

\subsubsection{Grundzüge und probleme der landwirtscahftlichen Sozialversicherung}

Aufgrund des strukturwandels ist die Situation der Solidaritätsversicherung schwierig, da die Anzahl der Einzahler schneller sinkt, als die Zahl der Bedürftigen.
Bei der \ac{LAK} (Pflichtversicherung für Landwirte) hat der Bund die Pflicht ein Defizit der \ac{LAK} über Steuergelder auszugleichen.
Ähnliches gilt für die landwirtscahftliche Krankenkasse.
Bei der landwirtschaftlichen Unfallversicherung leistet der Bund erhebliche Zuschüsse, hat aber keine Pflicht dazu.

Die Lebenserwartung ist ein guter Gradmesser für die Leistungsfähigkeit des Gesundheitssystems.
Allerdings steigen dadurch relativ stark die Ausgaben der Renten- und Krankenversicherung bei prozentual weniger Erwerbstätigen.

\subsubsection{Umlage und Kapiteldeckungsverfahren}

\paragraph{Umlageverfahren} besagt, dass Einzahlungen gleich den Auszahlungen einen Jahres sind.
Es können keine Rücklagen gebildet werden (Bzw nur schwer).
Eine Rettung des Umlageverfahrens wird auf Dauer nur über sehr hohe staatliche Zuschüsse möglich sein.
Die Zahlungsempfänger können, aufgrund ihrer großen Zahl, politische Forderungen stellen, wie zB eine Erhöhung der Renten.

Aufgrund der staatlichen Zuschüsse haben Landwirte eine zu hohe Rente (gemessen an der Einzahlung und Bevölkerung), was eine versteckte Subvention ist.

\paragraph{Kapitaldeckungsverfahren}
In diesem Verfahren wird das Geld angelegt und bekommt sein \glqq Geld\grqq{} wieder ausgezahlt.
Daher ist dieses Verfahren deutlich robuster gegenüber demographischen Veränderungen.

Da wir aktuell ein Umlageverfahren haben, bedeutet eine Umstellung auf ein Kapiteldeckungsverfahren eine Doppelbelastung der Einzahler.

\section{Produktionsfaktoren in der Landwirtschaft}

\subsection{Boden}
Der Produktionsfaktor Boden wird aktuell von der \ac{EU} im Rahmne der \ac{GAP} gefördert.
Ein Eingriff in den Erzeugermarkt über Stützpreise hatte sich als nicht effizient erwiesen.
In dem Marktdiagramm für den Faktor Boden, ist die Angebotskurve eine vertikale Gerade (komplett unelastische Mengen).
Die Nachfrage nach Boden orientiert sich an der \ac{WGP}.


%\begin{itemize}
%\end{itemize}
\end{document}
