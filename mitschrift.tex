\documentclass[11pt]{scrbook}
\usepackage[utf8]{inputenc}
\usepackage[ngerman]{babel}
\usepackage[a4paper, left=2cm, right=2cm, top=2cm]{geometry}
\usepackage[hidelinks]{hyperref}
\usepackage{acro}
\usepackage{amsmath}
\usepackage{cleveref}
\usepackage{todonotes}
\usepackage{longtable}
\usepackage{tabu}


\setcounter{secnumdepth}{4}
\setcounter{tocdepth}{4}

\acsetup{
	make-links ,
	index/use,
	trailing/activate = {dash}
}
%Acronyme definieren

\DeclareAcronym{VNS}{
	short = VNS,
	long = Vertragsnaturschutz,
}
\DeclareAcronym{BIP}{
	short = BIP,
	long = Bruttoinlandsprodukt,
}
\DeclareAcronym{UN}{
	short= UN,
	long = Vereinte Nationen,
}
\DeclareAcronym{ED}{
	short = ED,
	long = Einkommensdisparität,
}
\DeclareAcronym{LAK}{
	short = LAK,
	long = landwirtschaftliche Alterskasse,
}
\DeclareAcronym{WGP}{
	short = WGP,
	long = Wertgrenzproduktivität,
}
\DeclareAcronym{AK}{
	short = AK,
	long = Arbeitskraft,
}
\DeclareAcronym{GB}{
	short = GB,
	long = Großbritannien,
}
\DeclareAcronym{UdSSR}{
	short = UdSSR,
	long = Sowjetunion
}
\DeclareAcronym{CC}{
	short = CC,
	long = Cross Compliance,
}
\DeclareAcronym{EWR}{
	short = EWR,
	long = europäische Wirtschaftsraum
}
\DeclareAcronym{WTO}{
	short = WTO,
	long = Welthandelsorganisation
}
\DeclareAcronym{LEH}{
	short = LEH,
	long = Lebensmitteleinzelhandel
}
\DeclareAcronym{PSM}{
	short = PSM,
	long = Pflanzenschutzmittel
}

\DeclareAcronym{LCA}{
	short = LCA,
	long = Life-Cycle-Assesment
}
\DeclareAcronym{GAP}{
	short = GAP,
	long = gemeinsame Agrarpolitik,
}
\DeclareAcronym{UBA}{
	short = UBA,
	long = Umweltbundesamt
}
\DeclareAcronym{AG}{
	short = AG,
	long = Arbeitgeber,
}
\DeclareAcronym{VW}{
	short = VW,
	long = Volkswirtschaft,
}

\DeclareAcronym{BMEL}{
	short = BMEL,
	long = Bundesministerium für Ernährung und Landwirtschaft,
}

\DeclareAcronym{BWS}{
	short = BWS,
	long = Bruttowertschöpfung,
}

\DeclareAcronym{NWS}{
	short = NWS,
	long = Nettowertschöpfung,
}

\DeclareAcronym{EU}{
	short = EU,
	long = europäische Union,
}

\DeclareAcronym{VWL}{
	short = VWL,
	long = Volkswirtschaft,
}

\renewcommand*{\thesection}{\arabic{section}}

\title{Mitschrift APO WiSe 21/22}
\subtitle{Eine grauenhafte und unverständliche Mitschrift}
\author{Tibor Weiß}



\begin{document}
\maketitle
\newpage
\tableofcontents
\newpage
\printacronyms[
	heading = section*,
	name = Abkürzungsverzeichnis,
%	template = longtabu
]
\newpage

\addchap{Agrarpolitik}
\setcounter{section}{0}
\section{Einfürhung in die Agrarpolitik}
Beginn der VL!

\subsection{Grunddaten zur deutschen Land\-wirt\-schaft und Wert\-schöpf\-ung}
%\todo{VL hat mit nächsten Punkt angefangen}
\subsubsection{Statistische Daten}
Um Grunddaten über die deutsche Landwirtschaft zu erhalten, kann man beim \ac{BMEL}, statistisches Bundesamt oder Statista nach Daten suchen.
Diese veröffentlichen in der Regel Primärquellen.
Daten vom Deutschen Bauernverband (Lobbygruppe) sollte man nur mit viel Vorsicht verwenden, bzw. die Primärquellen ausfindig machen.

\subsubsection{Wie wichtig ist die Landwirtschaft an der dt. \ac{VW}}
\Ac{BWS} war 2020 der deutschen Landwirtschaft war 20,2 Mrd Euro, die \ac{NWS} lag bei 15,9 Mrd Euro.
Die \ac{BWS} ist der Produktionswert abzgl. der Vorleistungen.
\ac{NWS} ist die \ac{BWS} abzl. Abschreibungen und Abgaben und zzgl. Ausgleichszahlungen.


\todo{Vorlesungmitschrift aus fremden Mitschriften vervollständigen, hier ist eine Fehlstelle}
\subsection{Landwirtschaftliche Strukturwandel und Determinanten}

Was zeichnet den aktuellen Strukturwandel in der Landwitschaft aus und warum findet dieser statt.
Ist dies als Vor- oder Nachteil zu werten?

\subsubsection{Was ist der Strukturwandel?}
\begin{itemize}
	\item{Änderung von Daten im Sektor}
	\begin{itemize}
		\item Betriebsgröße
		
		\item Betriebsstrukturen

		\item Produktivität der Faktoren (Arbeit, Kapital, Boden)
	\end{itemize}
	\item{Bewertung von Strukturwandel (\ac{VW}-Sicht)}

		Generell ist Strukturwandel erwünscht, da so eine produktivere Wirtschaft ermöglicht wird. Dies führt dazu, dass \glqq schwache\grqq{}  Betriebe ausscheiden, da diese nicht in der Lage sind, ein nachhaltiges Einkommen zu erzielen.
Landwirtschaft ist der Strukturwandel sehr langsam, aufgrund der langen Investitionszyklen
	\item{Soziale Härten durch Anpassungen}

Immer weniger Betriebe im Velraufe der zeit (beobachtung)

\end{itemize}


\subsubsection{Determinanten}

Druck auf die Produzenten
\begin{itemize}
	\item{Spezialisierung und Skaleneffekte}

		Skaleneffekte nutzen, durchschnittliche Stückkosten senken (Spezialisierung bzw. Vergrößerung der Betriebe) und dadurch einen wirtschaftlichen Vorteil ggü anderen Betrieben erhalten.
	\item{technischer Fortschritt (auch biologisch, chemisch und organisatorischer)}

		Mög\-lich\-keit der Pro\-duk\-tivi\-täts\-stei\-ger\-ung, zB. durch Mechanisierung, Verbesserung der Sorten (Zucht), wirksame \ac{PSM} oder bessere Organisierung der Arbeitsabläufe (genauere Wettervorhersagen, vereinfachte Kommunikation).
		Dadurch erhöhen sich die Produktionsmengen und die Preise fallen (Marktdiagramm).
		Durch eine gesteigerte Nachfrage, kann der Effekt der fallenden Preise ausgeglichen werden, bzw verringert werden.
		In der landwirtschaftlichen Pri\-mär\-pro\-duk\-tion ist eine lokale Steigerung der Nachfrage häu\-fig nur über Be\-völ\-ker\-ungs\-wachs\-tum zu realisieren.
	\item{Außerlandwirtschaftliche Beschäftigungsmöglichkeiten}

		Abwanderung von Arbeitskräften, Notwendigkeit der Produktivitätssteigerung des Produktionsfaktors Mensch.
	\item{ungesicherte Hofnachfolge}

		Familienbetriebe (Von Inhaber geführte Betriebe) geben bei fehlendem Nachfolger (eigenes Kind) in der Regel auf (Beim Verkauf werden die Flächen in der Regel von anderen Betrieben gekauft und nicht von neuen Betrieben gekauft).
		Bei nicht vom Inhaber geführten Betrieben (Genossenschaften, AGs\ldots) wird ein neuer Verwalter, Betriebsleiter, Geschäftsführer\ldots von den Eigentümer eingestellt.
	\item{internationaler Wettbewerb}

		Produkte werden auf dem Weltmarkt gehandelt und beeinflussen daher die lokalen Preise.
		In den meisten Fäl\-len ist in einer Region die Produktion gün\-sti\-ger, sodass der Weltmarktpreis unter dem lokalen Preisgleichgewicht liegt.
		Sollte der Weltmarktpreis den lokalen Preis (deutlich) anheben, werden sich die Produktionsfaktoren verteuern (idR das knappste, derzeit Boden), da selbst mit einem höherem Faktorpreis ein Gewinn erzielt werden kann. 

		Aktuell (Herbst 2021) liegen die Weltmarktpreise für Getriede deutlich höher als die Produktionskosten.
		Wenn der Binnenmarkt am Welthandel teilnimmt, wird der Preis des Welthandels diktiert.
		Es entsteht eine Konsumentenrente (idR) oder Produzentenrente (aktuell) und führt zu Importen bzw. Exporten.

	\item{gesetzliche Auflagen}

	\item{Gesellschaftliche Anforderungen}

		82 Millionen Agrar-Experten in DE, welche unter dem Dunning-Kruger Effekt leiden.
	\item{kritische öff. Diskussion über die Landwirtschaft}
\end{itemize}

\subsection{Ziele der \ac{EU} Agrarpolitik und der deutschen Agrarpolitik}
%Beginn VL am 26.10.21
Warum werden Ziele (in der Politik) ungenau formuliert?
\begin{itemize}
	\item Um nicht an den Zielen gemessen werden zu können (Scheitern/Benotung uä)

		Schröder hatte damals eine bestimmte Arbeitslosenquote versprochen, dies wurde von den Medien ausgenutzt.

		Künast hatte eine Quote von 20\% Anbaufläche von ökologischer Landwirtschaft als Ziel genannt.
		Mit 7\% wurde das Ziel deutlich verfehlt, dies konnte entsprechend instrumentalisiert werden.

	\item Ziele der agrarpolitischen Entscheidungsträger sind (wahrscheinlich) ungleich zu denen der Agrar- und Umweltpolitik

		Politiker müssen in der Regel ihr Wählerstimmen maximieren - Landwirte/landwirtschaftlich nahe Unternehmen sind ein relativ kleines Wählerpotential.
		Künast hatte das Landwirtschaftsministerium um Verbraucher erweitert, um das Wählerpotential der Grünen zu erweitern.

	\item Viele Umwelt- und Agrarpolitische Maßnahmen haben einen sehr langen Zeithorizont

		Eine Veränderung der Düngung von landwirtschaftlichen Flächen hat einen Effekt auf die Nitratproblematik, allerdings sind diese erst nach 10 Jahren zu erkennen.

	\item Produktivitätssteigerung und bessere Stellung der Landwirtschaft ggü anderen wirtschaftlichen Branchen
		\todo{Vorlesung am 26.10.21 verlassen}
\end{itemize}

Zielkonflikte in der landwirtschaftlichen Agrarpolitik entstehen regelmäßig.
Es gibt viele Ziele, welche in direkter Konkurrenz zueinander stehen, zB CO\textsubscript{2}-Minderung versus Tierwohl.
CO\textsubscript{2}-Minderungen über Stilllegung von Mooren steht im Konflikt mit der Einkommenspolitik in der Agrarpolitik. Dies könnte man zB über Entschädigungen entschärfen.

Zukünftig werden diese und weitere Zielkonflikte die Agrarpolitik dominieren.
In der breiten Bevölkerung (und tlw auch Politik) fehlt häufig das fachliche Verständniss um solche Abschätzungen bewerten zu können.

\subsection{Gründe für staatliche Eingriffe in den Agrarsektor}

\paragraph{Marktversagen} \todo{Wiederholung VL am 26.10.21}
\begin{itemize}
	\item Ressourcenverteilung
		Bei externen Effekten sollte der Staat eingreifen, um sicherzustellen, dass die Ressourcen möglichst effizient verteilt werden.

	\item Viele Umweltgüter haben an sich keinen Preis.
Aufgrund der Martwirtschaft werden diese Güter relativ stark nachgefragt werden.
Die Politik muss dann entsprechend Regeln über Preise (CO\textsubscript{2}-Preise) in den Markt eingreifen.

\item Informationsassymetrie
	
	siehe Vorlesung Produktqualität

\item Marktmacht

\item Bei einem technisch vernünftigen Marktergebnis können soziale Probleme entstehen.
	Landwirte haben systematische Beeinträchtigung über die Risiken des Wetters.
\end{itemize}

Marktversagen rechtfertigt einen Eingriff der Politik.


\section{Landwirtschaftliche Einkommens- und Sozialpolitik}
\todo{hier Ende VL 26.10.21}
\subsection{Einkommensdisparität der Landwirtschaft}
\acl{ED}
\begin{itemize}
	\item innere
	\item äußere
\end{itemize}

\subsubsection{Testbetriebsnetz}
Das Testbetriebsnetz wird genutzt, um die \ac{ED} zu ermitteln.
Es wird nicht nur das Einkommen erfasst, sondern auch viele sekundäre Daten wie Finanzierung oder Ausstattung mit AK.
Die Veröffentlichung erfolgt im \glqq Agrarpolitischen Bericht der Bundesregierung\grqq{}, wobei die Ergebnisse auf die Grundgesamtheit in Deutschland hochgerechnet werden.

Für die Vergleichsrechnung werden nur Haupterwerbsbetriebe an.
\begin{itemize}
	\item Der Gewinn wird für die Entlohnung der Arbeit und die eingesetzten Produktionsfakoren.

	\item Der Betriebsinhaber wird der durchschnittliche Bruttolohn (ohne \ac{AG}-Anteil) angesetzt. Für mitarbeitende Familienangehörige inkl. \ac{AG}-Anteil.

	\item Betriebsleiter wird abhängig vom Umsatz einen höheren Lohn bekommen. Kalkulatorisch mit 7€ mehr Lohn je 1000€ mehr Umsatz

	\item Zinsansatz für das eingesetzte Eigenkapital

	\item Ziel ist einen \glqq durchschnittlichen Selbständigen\grqq{} zu repräsentieren.
\end{itemize}

Im Agrarpolitischen Bericht wird der Unterschied in \% angegeben um die äußere \ac{ED} darzustellen.
In den meisten Jahren, ist dieser Prozentsatz negativ, dies zeigt, dass viele Betriebe zu teuer wirtschaften.
Häufig stehen größere Betriebe deutlich besser da, als kleinere Betriebe.

\subsubsection{spezielle Probleme der Vergleichsrechnung} existieren bei vergleichen zwischen den Jahren, muss man berücksichtigen, dass sich das Testbetriebsnetz ständig verändert.
Land wird in der Regel zum Anschaffungswert und nicht zum Verkehrswert gerechnet.
Aufgrund der steigenden Bodenpreisen wird dies nicht berücksichtigt, somit ist der Zinsansatz deutlich unterschätzt.
Außerlandwirschaftliches Einkommen wird nicht betrachtet.
Des weiteren wird das Bruttoeinkommen und nicht das Nettoeinkommen verglichen.

\subsubsection{verfügbares Einkommen} ist für die Einschätzugn der sozialen Lage der Landwirtschaft ist eine Berechnung des verfügbaren Einkommens ein guter Ansatzpunkt.
Dabei werden alle Einkommen aller Familienmitglider (Haushalt) nach den Steuern und sozialen Abgaben berechnet.

Eine \ac{ED} zu anderen, vergleichbaren Haushalten ist (häufig) nicht zu erkennen.
Allerdings werden Altenteileraufwendungen und ähnliche Aufwendungen nicht berücksichtigt.
Des weiteren sind in landwirtschaftlichen Haushalten häufig größer (mehr Kinder).

\subsection{Landwirtschaftliche Sozialpolitik}

\subsubsection{Prinzipien der Sozialpolitik}

\begin{itemize}
	\item Äquivalenzprinzip - mehr Einzahlung, mehr Leistung

		hohe Bereitschaft der Versichungsnehmer ihre eigene wirtschaftliche Stellung zu verbessern
	\item Solidaritätsprinzip - mehr Einzahlung - gleiche Leistung, zB Krankenversicherung, Beitrag abhängig vom Lohn, trotzdem gleiche Leistung

		Eigenverantwortung und Leistungsbereitschaft werden nicht belohnt oä, daher gering und tendenz zur Rationalitätsfalle zur Ausweitung der Leistungsausbreitung und dadurch eine Beitragsleistungsausweitung

	\item Versorgungsprinzip - keine Beiträge, aber Leistung, zB Hartz IV

		Wird nur verwendet, wenn nicht anders möglich!
	\item Subsidaritätsprinzip - Selbsthilfe, Hilfe aus dem nahen Umfeld, zB Versicherungen mit Selbstbeteiligung
\end{itemize}

\subsubsection{Grundzüge und probleme der landwirtscahftlichen Sozialversicherung}

Aufgrund des strukturwandels ist die Situation der Solidaritätsversicherung schwierig, da die Anzahl der Einzahler schneller sinkt, als die Zahl der Bedürftigen.
Bei der \ac{LAK} (Pflichtversicherung für Landwirte) hat der Bund die Pflicht ein Defizit der \ac{LAK} über Steuergelder auszugleichen.
Ähnliches gilt für die landwirtscahftliche Krankenkasse.
Bei der landwirtschaftlichen Unfallversicherung leistet der Bund erhebliche Zuschüsse, hat aber keine Pflicht dazu.

Die Lebenserwartung ist ein guter Gradmesser für die Leistungsfähigkeit des Gesundheitssystems.
Allerdings steigen dadurch relativ stark die Ausgaben der Renten- und Krankenversicherung bei prozentual weniger Erwerbstätigen.

\subsubsection{Umlage und Kapiteldeckungsverfahren}

\paragraph{Umlageverfahren} besagt, dass Einzahlungen gleich den Auszahlungen einen Jahres sind.
Es können keine Rücklagen gebildet werden (Bzw nur schwer).
Eine Rettung des Umlageverfahrens wird auf Dauer nur über sehr hohe staatliche Zuschüsse möglich sein.
Die Zahlungsempfänger können, aufgrund ihrer großen Zahl, politische Forderungen stellen, wie zB eine Erhöhung der Renten.

Aufgrund der staatlichen Zuschüsse haben Landwirte eine zu hohe Rente (gemessen an der Einzahlung und Bevölkerung), was eine versteckte Subvention ist.

\paragraph{Kapitaldeckungsverfahren}
In diesem Verfahren wird das Geld angelegt und bekommt sein \glqq Geld\grqq{} wieder ausgezahlt.
Daher ist dieses Verfahren deutlich robuster gegenüber demographischen Veränderungen.

Da wir aktuell ein Umlageverfahren haben, bedeutet eine Umstellung auf ein Kapiteldeckungsverfahren eine Doppelbelastung der Einzahler.

\section{Produktionsfaktoren in der Landwirtschaft}

In der Landwirtschaft gibt es viele Besonderheiten bei den Faktormärkten.

\paragraph{Exkurs \ac{WGP}:}
Der Faktorpreis für einen Faktor wird durch die \ac{WGP} beschränkt, bzw. häufig davon bestimmt.
Die \ac{WGP} gibt an, wieviel finanziellen Ertrag ein Betrieb über die Zunahme von einer minimalen Einheit des Faktors.
Preiserhöhung auf den Produktmärkten ändert sich die \ac{WGP}-Kurve - somit ändert sich die Zahlungsbereitschaft.

\subsection{Boden}
Der Produktionsfaktor Boden wird aktuell von der \ac{EU} im Rahmne der \ac{GAP} gefördert.
Ein Eingriff in den Erzeugermarkt über Stützpreise hatte sich als nicht effizient erwiesen.
In dem Marktdiagramm für den Faktor Boden, ist die Angebotskurve eine vertikale Gerade (komplett unelastische Mengen).
Die Nachfrage nach Boden orientiert sich an der \ac{WGP}.
Boden ist nicht vermehrbar und Deutschland hat schlechte Erfahrung damit gemacht, weitere Ländereien zu Deutschland hinzufügen.

Durch die Preissteigerung von Boden kann bei der Verwendung von Boden als \glqq Anlageobjekt\grqq{} eine höhere \ac{WGP} erreicht werden.
Bei einer Reduzierung der verfügbaren landwirtschaftlichen Fläche (zB durch Stilllegung oder ökologischen Vorrangsfläche, \glqq Flächenverbrauch\grqq{}) wird der Faktor Boden knapper.
Dadurch verschiebt sich die Angebotskurve nach links und somit steigt der Faktorpreis und die Betriebe, welche im Grenzbereich unterwegs waren, werden wirtschaftlich zur Aufgabe ihrer Unternehmung gezwungen (mittelfristig)

Auf den Bodenmarkt haben Preisänderungen am Produktmarkt relativ große Einflüsse.
Da die Menge des genutzten Bodens durch höhere oder niedrigere Preise nicht verändert werden kann, fallen die Änderungen im Faktorpreis relativ deutlich aus.

\subsubsection{Direktzahlungen} 

Die Direktzahlungen erhöhen die \ac{WGP}, indem die \ac{WGP}-Kurve nach oben verschoben wird.
Damals wurden die Garantiepreise für Agrarprodukte abgeschafft, bzw. deutlich abgesenkt.
Dadurch sinkt die \ac{WGP} des Bodens.
Über die Direktzahlungen sollten die gesunkenen Produktpreise ausgegelichen werden.
Allerdings wurden die gesunkene \ac{WGP} des Faktors Boden überkompensiert, da die Direktzahlungen auch die gesunkene \ac{WGP} der Faktoren Arbeit und Kapital abfangen sollten.
Dieser Effekt tritt auf, da die Direktzahlungen direkt an die landwirtschaftliche Fläche gekoppelt sind.

Somit wurde der Faktormarkt Boden aus dem Gleichgewicht gebracht und die Bodenpreise mussten dann entsprechend stark steigen.
Des weiteren werden die Faktoren Arbeit und Kapital deutlich schlechter entlohnt.

Die Bodenpreise (Kauf) in Deutschland haben sich von 2007 auf 2017 um den Faktor 2,5 erhöht.
Bei der Osterweiterung der \ac{EU} haben die neuen Staaten einen langsamen Einstieg in die Direktzahlungen bekommen.
In den Altländern der \ac{EU} wurden die Preise langsam abgesenkt, um den \ac{EU}-Haushalt nicht aus dem Gleichgewicht zu bekommen.

\subsubsection{Andere Effekte} 
Es gibt weitere Effekte welche zur Preissteigerung auf dem Faktormarkt Boden führen, zB Biogasanlagen, welche von der Politik stark gefördert wurden.
Auch Auflagen wie geringere Ausbringmenge von organischen Düngern oder weitere Fruchtfolgen werden den Effekt der steigenden Bodenpreise weiter verstärken (Bodenverknappung).
Es exisiteren auch Spill-over Effekte von großen Städten oder die Gefahr von Spekulationsblasen.

\subsubsection{Spekulationsblasen}
Es gibt Spekulation auf dem (deutschen) Landmarkt.
Bisher hat es aber keine (große) Blase gegeben und die Preise waren im Großen und Ganzen durch die \ac{WGP} gedeckt.
Über eine schnelle Einstellung der Direktzahlungen über die \ac{GAP} würde eine Blase entstehen.

\subsubsection{Schlussfolgerungen}

\begin{itemize}
	\item Boden ist Vermögensgegenstand mit komplexen Verhältnissen
	\item Kau- und Pachtpreise (Zinsen) sollten für eine landwirtschaftliche Nachfrage relativ konstant sein
	\item Hohe Bodenpreise, bzw. starke Steigerung der Bodenpreise ist nicht neu und über die \ac{GAP} zu erklären
	\item Der Kauf von Boden ist relativ teuer und für landwirtschaftliche Betriebe in der Regel nicht wirtschaftlich darstellbar.
		Aufgrund der Steigerung des Wertes bei Ausweisung eines Baugebietes ist eine Spekulation, aber ein Grund dafür, Land zu kaufen, um davon zu profitieren.
	\item Die große Varianz in der Kauf- Pachtpreisrelation zwischen den Bundesländern ist über verschiedene Effekte zu erklären.
		So ist in den neuen Bundesländern eine deutlich höhere Aktivität auf dem Bodenmarkt, da viele Agrargenossenschaften auch größere Landflächen an außerlandwirtschaftliche Investoren verkaufen.
		Die kleinstrukturierten Regionen wie zB Bayern sind für Investoren nicht so attraktiv.
\end{itemize}

\subsection{Arbeit}
Bei einer höheren Entlohnung würden mehr \ac{AK} in die Landwirtschaft kommen.
Somit würde bei einer Anhebung der Löhne auch mehr Personal in die Landwirtschaft kommen.

Die Bedeutung des landwirtschaftlichen Arbeitsmarkt sinkt, da die Effektivität einzelner \ac{AK} deutlich steigt.
Da die Fläche nicht steigt, sinkt somit de absolute Zahl der \ac{AK}.
Des weiteren sind viele \ac{AK} aus der Familie oder nur für Saisonarbeiten eingestellt.
Somit ist der Markt für andere \ac{AK} sehr klein.
10\% der Betriebsleiter haben ein abgeschlossenes Hochschulstudium.

Insbesondere die Familen-\ac{AK} sind in der Ausgestaltung ihrer Arbeitszeit relativ frei und können weniger Freizeit für mehr Einkommen (oder andersherum) substituieren.
So kann in Phasen mit geringen Produktpreisen über mehr Arbeit die wirtschaftliche Lage ausgeglichen werden.
Mit steigendem Anteil an Nichtfamilien-\ac{AK} fehlt diese Möglichkeit auf schlechte Preise zu reagieren.

Durch das Ausweiten der Produktion bei sinkenden Preisen wird der Effekt der sinkenden Preise weiter angetrieben.
Somit kann sich die Landwirtschaftliche Branche selbst in eine größere Preisdepression drücken.
Aufgrund der langen Produktionszyklen (ein- bis mehrjährig, ausgenommen Schweineproduktion) kann auf Preisänderungen nicht schnell reagiert werden.
Diese Situation ist mit der Spieltheorie (Gefangenendilemma) zu erklären.
Ein Langfristiger Trend kann aus dieser inversen Reaktion allerdings nicht folgen.

Dem Preisdruck kann man sich dadurch anpassen, dass nach wirtschaftlichereren Technologien umgesehen wird.
Dies hätte aber bereits vorher geschehen können und kann sonst zu einem inversen Angebotsverhalten führen.

\subsection{Kapital}
Auch bei einem deutlich höheren Kapitalbedarf im Landwirtschaftlichen Sektor wäre der globale Kapitalmarkt nicht aus der Balance.
Somit sind die Kosten von Kapital für den landwirtschaftlichen Sektor absolut elastisch.
Aufgrund des hohen Investitionsbedarfs (ca. 500.000€/\ac{AK}) ist Kapital für die Landwirtschaft sehr wichtig.

\subsubsection{Zinssatz}
Der Zins beschreibt die Kosten des Kapitals.
Es gibt immer wieder Förderungen für Landwirte, über welche der landwirtschaftliche Kapitalmarkt sehr günstig Kapital bereit stellen kann.
Die Beleihung des Bodens über Grundschuld ist für die Landwirte eine gängige Möglichkeit den Zinssatz zu senken.
Über diese Zinsverbilligung wird in der Landwirtschaft mehr Kapital eingesetzt.

\section{\ac{EU}-Agrarpolitik: Grundlagen zu den Entscheidungsinstanzen und zum Haushalt}

Der \ac{EU}-Binnenmarkt ist (gemessen am \ac{BIP}) der größte gemeinsame Markt der Welt.
Afrika versucht die \ac{EU} zu kopieren (im Bezug auf den Handelsverbund), daher ist der Einfluss auf Afrika nicht zu unterschätzen.
Heutzutage hat die \ac{EU} eigene Persönlichkeitsrecht bei der \ac{UN}, sowie deren Mitglieder.

Gründungshintergrund war im Angesicht des Krieges, dass der Wiederaufbau und die Versorgung mit Nahrungsmitteln in Europa sicher zu stellen.
Über den Aufbau von Handelsbeziehungen, sollte verhindert werden, dass sich militärische Konflikte entwickelt können.

1995 wurde die \ac{EU} um 3 Länder auf 15 Länder erweitert - heutzutage als \glqq alte\grqq{}-\ac{EU} bezeichnet.

2004 und 2007 wurde die \ac{EU} um zwölf ehemalige \ac{UdSSR}-Länder Mitglied der \ac{EU}.
Durch den Austritt von \ac{GB} in 2020 könnte es zu einem Referndum in Schottland kommen, sodass die Schotten sich unabhängig machen um sich der \ac{EU} anzuschließen.

Aufgrund der Zollunion sind die Länder sehr eng aneinander gebunden.
Ein Eintritt in die \ac{EU} verändert somit die komplette Handelsbeziehungen (ausgehandelte Zölle) des neuen Mitgliedsstaates.

In der \ac{EU} wurde die Aussenhandlungsbeziehungen und die Agrarpolitik zentralisiert.
In einigen Ländern wurden eine gemeinsame Währung eingeführt.
Generell werden Stück für Stück weitere Kompetenzen von den Mitgliedsstaaten auf die \ac{EU} übertragen.
Ein wichtiges Ziel ist, gemeinsam gegenüber Drittstaaten aufzutreten.

Der EURO hat zwar einen sehr großen Handelsraum, allerdings hat der US-Dollar deutlich gelitten, auch wenn der US-Dollar weiterhin für viele internationale Märkte eingesetzt.
Allerdings ist der EURO eine stabile Währung und  neben dem US-Dollar die wichtigste Währung, trotz ihres geringem Alters.

\subsection{Grundlagen der \ac{GAP}}

Der Austritt von \ac{GB} aus der \ac{EU} hat die Position von der deutschen Meinung in der \ac{GAP} geschwächt, da die politische Forderungen/Meinungen häufig sehr ähnlich waren.

Das Ziel der Gründung der \ac{EWR} war einen ungehinderten Warenaustausch zwischen den Mitgliedsstaaten.
Viele lokale Richtlinien/Traditionen verhindern eine schnelle Einführung.
Das Ziel wurde, im Allgemeinen, erst in 1992 mit der Schaffung des \ac{EU}-Binnenmarktes erreicht.

\subsubsection{Grüne Kurse}
Für eine gemeinsame Agrarpolitik wurden Interventionspreise festgelegt.
Da die Mitgliedsstaaten eigene Währungen hatten, wurde eine virtuelle Währung erschaffen, in der die Interventionspreise angegeben.
Die Wechselkurse wurden dann für den Agrarsektor verändert, um die eigenen Ziele zu unterschützen.

Als diese Möglichkeit der grünen Kurse abgeschafft hat, wurde der Mehrwertsteuersatz der pauschalierenden Betriebe erhöht.

\subsubsection{Reinheitsgebot Bier}
Zum Beispiel das deutsche Reinheitsgebot für Bier steht in Konflikt mit dem freien Markt.
Entweder muss die komplette \ac{EU} das Reinheitsgebot übernehmen oder Deutschland muss das Reinheitsgebot fallen lassen.
Deutschland musste das Reinheitsgebot fallen lassen.

Allerdings hat sich das deutsche Reinheitsgebot für Bier als Qualitätsmerkmal international durchgesetzt.
In Deutschland finden Biere, welche nicht nach dem deutschen Reinheitsgebot gebraut wurden, kaum Absatz.

Andere privatwirtschaftliche Systeme (zB Tierwohllabel) welche durch den \ac{LEH} im Markt durchgesetzt werden, sorgen dafür, dass sich regionale Märkte bilden.

\subsubsection{Gemeinschaftspräferenz}
\ac{EU}-Produkte müssen vorrang vor importierten Produkten haben.
Dadurch sollten die Preise auf den heimischen Märkten zu höheren Preisen als den Weltmarktpreisen.
Dies ist als Außenschutz zu verstehen.

Dies entspricht aber nicht den Zielen der \ac{WTO}.
Die \ac{WTO} will möglichst geringe Zölle zwischen den Ländern, da dies der Weg ist, für einen hohen Lebensstandard auf der Welt zu sorgen.
Somit werden auch die Drittstaaten von der \ac{EU} immer besser behandelt, bzw. entsprechende Handelsabkommen abgeschlossen.

\subsubsection{Direktzahlungen}
Im Rahmen der \ac{GAP} während der Osterweiterung (ab 2004) wurden unterschiedliche Direktzahlungen je ha eingeführt.
In den Ostländern, wurden diese mit geringen Beträgen eingeführt und werden an das Niveau der alten Staaten angeglichen.
Daher werden in Deutschland die Direktzahlungen weiter sinken werden.

\subsection{Entscheidungsinstanzen der \ac{EU}-Agrarpolitik}

\subsubsection{Die \ac{EU}-Kommission}
Die Kommissionsmitglieder werden für jewiels 5 Jahre von den Mitgliedsregierungen ernannt.
Jedes \ac{EU}-Mitglied stellt einen Kommissar. Ein Kommissar sind diese mit Minister zu vergleichen.
Die Präsidentin der \ac{EU}-Kommission ist vergleichbar mit einem deutschen Bundeskazler, hat allerdings etwas weniger Rechte.

Natürlich sollen die Mitglieder im Interesse der Gemeinschaft der \ac{EU} handeln.
Die Mitglieder sind nicht an die Weisungen ihrer nationalen Regierung gebunden.
Das europäische Parlament hat im Allgemeinen ein Vetorecht.

Die Kommission hat Exekutivrecht, ist für die Durchsetzung der beschlossenen \ac{EU}-Gesetze zuständig.
Dafür können Durchführungsverordnung von der Kommission erlassen werden.
Dadurch können sehr viele Detailfragen sehr genau geregelt werden.

Die Kommission ist das einzige Organ der \ac{EU}, welche Gesetze Vorschlagen darf (Initiativrecht).
Dadurch hat der Ministerrat nur eingeschränkten Einfluss auf die Gesetzgebung.
Allerdings muss der Ministerrat den Gesetzen zustimmen.


\subsection{Haushalt der \ac{EU}-Agrarpolitik}
\todo{hier fehlt etwas}


\section{Instrumente der Agrarpolitik}

\subsection{Beurteilungskriterien}

\begin{itemize}
	\item Preis- und Mengeneffekte von Politikmaßnahmen
	\item Ordnungspolitische Einordnung von Politikinstrumente
	\item Wohlfahrtseffekte
	\item Verteilungseffekte
	\item Budgetäre Effekte
	\item Effekte auf den internationalen Handel
		
		Eine Hochpreispolitik über die Interventionspreise haben die Weltmarktpreise sehr stark unter Druck gesetzt

	\item Administrative Durchführbarkeit
		
		Bei Direktzahlungen ist dies eine gewisse Herausforderung, da alle Flächen genau erfasst werden müssen

	\item ökologische Effekte
		werden immer wichtiger, müssen heutzutage mit einbezogen werden, auch wenn es sehr schwierig ist.
	\item Tierwohleffekte
\end{itemize}


\subsubsection{Auswirkungen am Beispiel Exportsubvention}

Ausgangssituation ist dieser, dass der Marktpreis der \ac{EU} deutlich über dem, des Weltmarktes ist.
Des weiteren ist die \ac{EU} Exportland, ergo müssen die Anbieter ihre Waren auf dem Weltmarkt verkaufen.
Um die Anbieter zu unterstützen, werden für Exporte Subventionen bezahlt.
Sonst würde die \ac{EU} in den eigenen Agrarprodukten ersticken.


\begin{itemize}
	\item P\textsubscript{EU}= Marktpreis der \ac{EU}
	\item P\textsubscript{W}= Weltmarktpreis
	\item Q\textsubscript{N}= Nachfragemange
	\item Q\textsubscript{A}= Angebotsmenge
\end{itemize}


\begin{itemize}
	\item Preis- und Mengeneffekte von Exportsubventionen

		Dadurch, dass P\textsubscript{EU} deutlich erhöht ist, steigt Q\textsubscript{A}, gleichzeitig sinkt Q\textsubscript{N}.
		Daher steigt der Marktüberschuss überproportional an.
		Dieser muss Exportiert werden, und die muss über Exportsubventionen bezahlt werden, um die Preise in der \ac{EU} gleichzuhalten.
		Dies senkt indirekt den P\textsubscript{W}.

	\item Wohlfahrtseffekte der Exportsubvention
	
\begin{itemize}
	\item $\Delta$Konsumentenrente = -A -B (erhöhte P\textsubscript{EU})
	\item $\Delta$Produzentenrente = +A +B + C (erhöhte P\textsubscript{EU})
	\item $\Delta$Staat für Exportsubvention = -B -C -D
	\item $\Delta$Wohlfahrt für die Volkswirtschaft = -B -D
		

\end{itemize}
		Also entsteht bei einer Exportsubventionierung ein negativer Wohlfahrtseffekt und somit für die Volkswirtschaft als Nachteilig zu bewerten.

	\item Verteilungseffekte
	\item Budgetäre Effekte
	\item Effekte auf den internationalen Handel
	\item Administrative Durchführbarkeit
	\item ökologische Effekte
	\item Tierwohleffekte
\end{itemize}

\subsection{Kapitel 5.2}
\subsection{Kapitel 5.3}
\todo{Mitschrift bis zum 3.12 fehlt}
%Kapitel 5.4
\subsection{Mengensteuerungspolitiken bei Agrarprodukten}

Über eine Mengenreduzierung können die Preise gestützt werden.
Die Menge wird dann über eine Quote geregelt.
Somit kann auf dem Binnenmarkt ein höherer Preis als auf dem Weltmarkt existieren, ohne dass es einen Exportüberschuss gibt.
Wenn die Qoute gehandelt werden kann, ist diese entsprechend teuer, da der Marktpreis deutlich über den Produktionskosten ist.

Bei der Einführung der Qoute bei der Milchmenge wurden Ausnahmeregelungen von Landwirten sehr kreativ ausgenutzt, um eine möglichst geringe Mengenreduzierung zu bekommen.
Dadurch, dass die Mengenreduzierung nicht in dem Umfang stattgefunden hat, wie erwartet, hat sich kein ausgeglichener Binnenmarkt eingestellt.
Somit musste weiterhin Milch mit Exportzuschüssen exportiert werden, auch wenn deutlich geringere Mengen.
Des weiteren steigt der P\textsubscript{Weltmarkt} und somit reduziert sich die Exportsubvention je Einheit.
Somit war der Schaden für die Volkswirtschaft geringer.

Heutzutage gibt es innerhalb der \ac{EU} nur in weingen bestimmten lokalen Märkten mit Mengenregelungen.

Die Wohlfahrt bei einer Einführung von Qouten bei exisitierenden Garantiepreisen ist für die \ac{VWL} positiv, da die negative Effekte der Garantiepreise verringert werden.
Allerdings entstehen sehr hohe Aufwendungen für die Administrative Seite (Kosten für den Staat) und auch für die Betriebe, da diese auch bestimmte Dokumentation/Anträge und ähnliches bearbeiten muss.

Bei der Möglichkeit des Handels der Qouten, führt es dazu, dass Betriebe, welche aus der Produktion aussteigen, bevorteilt werden und für diejenigen, welche ihre Produktion ausweiten, benachteiligt.
Somit wird der status quo verfestigt und es werden kaum Degressionseffekte über steigende Mengen bei den Betrieben realisiert.

\subsubsection{Markteinschränkungen}
Langfristig entstehen entprechende Kosten in der Verwaltung auf den entsprechenden Ebenen (Land bis \ac{EU} bei der Milchquote).
Um die Mengenbeschränkungen auf den lokalen Märkten aufrechtzuerhalten, muss der Binnenmarkt vom Weltmarkt abzukoppeln.
Dies steht im Konflikt mit vielen Handelsabkommen und den Zielen der \ac{WTO}.
Die Handelsabkommen können somit nicht erfüllt werden, bzw. es müssen den anderen Handelspartnern auch Vorteile geben werden.
Bei solchen Einschränkungen ist davon auszugehen, dass diese nicht ohne negative Auswirkungen auf die internationalen Beziehungen möglich ist.

\subsubsection{Strukturwandel}
Über den Verkauf der Produktionsrechte, wird Wachstum für Betriebe sehr teuer und aussteigenden Betriebe werden mit Geld belohnt.
Dadurch wird der Strukturwandel verzögert.

Des weiteren werden, aufgrund der Reduzierung einer Produktion, andere Produktionszweige ausgeweitet.
Dies führte beim einführen der Milchquote zu einer Erhhöhung der Schweinefleichproduktion und somit ein Verfall des Schweinepreises.

Quotenkürzungen sind politisch kaum durchsetzbar.
Wenn die Quote zu hoch ist, muss die Überproduktion weiterhin teuer am Weltmakrt verkauft werden.

\subsubsection{Fazit}
Mengenregelungen sind nicht per Gesetz durchsetzbar.
Freiwillige Programme, wie von der \ac{EU}, sind allerdings möglich.
An solchen Programmen, nehmen nur die Beriebe teil, welche einen geringen Deckungsbeitrag haben und somit langfristig vermutlich nicht wirtschaftlich am Markt besetehn würden.


\subsection{Direktzahlungen an Agrarprodukten}
\subsubsection{EU-Agrarpolitik seit 1992 (MacScherry)}
Bis 1992 gab es in der \ac{EU} eine Hochpreispolitik bei den Agrarprodukten.
Die hohen Mindestpreise konnten finanziell von der \ac{EU} nicht weiter gestemmt werden.

Daher wurden die Mindestpreise um 35\% gesenkt (Bei Getreide und Ölsaaten).
Um den Produzenten einen Ausgleich zu zahlen, wurde der Verlust des durchschnittliche Deckungsbeitrages und entsprechende Einkommensreduzierung.
Diese Einkommensreduzierung wurden über die Direktzahlungen eingeführt, damals unter dem Namen Preisausgleichszahungen.

1992 wurde auch eine Möglichkeit der Flächenstilllegung geschaffen, um die Produktionsmenge zu reduzieren.

\subsubsection{Agende 2000}
Mindespreissenkungen wurden als sinnvoll bewertet, die Direktzahlungen wurden in dem Zug auch als sinnvoll eingestuft.
Daher sollte das Programm weiter geführt werden und eine Übetragung auf den Milchmarkt wurde auch als sinnvoll eingestuft.
Es wurde entsprechend die \ac{GAP} eingeführt, um die ländliche Räume zu fördern.

\subsubsection{Halbzeitbewertung 2003}
Auf dem Milchmarkt gab es bereits eine Milchquote (Einführung 1984), um die staatlichen Mindestpreise halten zu können.
Da die Quote zu hoch war, mussten 2003 die Mindestpreise deutlich abgesenkt werden - was im Zuge der Reform von 1992 als nicht notwendig hingestellt wurde.
Die Interventionspreise für Butter und Magermilchpulver wurden abgesenkt, und darüber indirekt der Milchpreis.

Die Ausgleichszahlungen wurden anfäglich an den Quoten festgemacht. Später wurden diese von der Produktion entkoppelt und an die Fläche gekoppelt.
Um \ac{CC}-Auflagen einzuhalten, wurden die Zahlungen auf zwei Säulen aufgeteilt, wobei die Zahlungen aus der zweiten Säule an Auflagen gekoppelt waren.

Butter und Magermilchpulver wurden unterschiedlich abgesenkt.
Die Magermilchpulverpreise wurden nicht so stark abgesenkt, da dieser nicht so weit von den Weltmarktpreisen entfernt war, wie bei der Butter.

\subsubsection{Healthcheck 2008}
Über eine Anhebung der Quoten werden diese schleichend entwertet.
Dies erleichtert den Ausstieg aus dem Quotensystem.

\subsubsection{GAP 2013}
Für den Wirkungszeitraum 2014-2020 wurden die Direktzahlungen weiter gekürzt.
Es sollten kleinere Betriebe stärker bevorzugt werden und es wurden auflagen für die Direktzahlungen eingeführt, heute bekannt unter \textit{Greening}.
Des weiteren wurden die Zahlungen von der ersten in die zweite Säule umgeschichtet, um zB ökologische Produktion zu fördern.

\subsubsection{GAP 2020}
Im großen und ganzen wurde die politik der \ac{GAP}-Reform von 2013 weiter fortgeführt.
Es wurden \textit{Eco-Schemes} eingeführt.
Aufgrund von Verzögerungen werden diese Änderungen erst 2023 zum Tragen kommen.
Das Ende des Wirkungszeitraum wird allerdings nicht verschoben - zumindestens ist das so geplant.

\subsubsection{GAP 2027}
In 2027 ist die nächste Reform fällig, da die letzte Reform zwei Jahre zu spät ist.

\subsubsection{Wirkung der Reduzierung der staatlichen Mindestpreise und Einführung von Direktzahlungen}
\label{subsubsec:mindest}
Aufgrund der sinkenden Preise steigt die Nachfrage nach Agrarprodukten.
Aufgrund der Bedingungen der damaligen Direktzahlungen, hatten die Landwirte keinen Anreiz ihre Produktionsmenge zu reduzieren.
Die steigende Nachfrage sorgt dafür, dass die Exportmenge reduziert wird.
Dadurch steigt der Weltmarktpreis, da die \ac{EU} ein großer Aktuer auf dem Weltmarkt ist.

Die Konsumentenrente steigt etwas an (gesunkene Preise und höhere Mengen).
Bei den Produzenzen verlieren zwar durch die Preissenkung einen Teil ihrer Produnzentenrente.
Die Kompensierung war allerdings passend dafür.
Der Staat muss zwar die Direktzahlung/Prämie bezahlen.
Dafür muss eine deutlich geringere Menge exportiert werden mit entsprechenden Exportsubventionen.
Durch den höheren Weltmarktpreis reduzieren sich die Differenzen zwischen den Märkten, sodass die staatliche Ausgaben mehr als kompensiert werden.

Also hat der Staat deutliche Wohlfahrtsgewinne realisiert, genauso wie die Landwirte und die Konsumenten.
Es enstehen also positive Wohlfahrtseffekte.

\subsubsection{Entkopplung der Direktzahlungen}
Durch eine Entkopplung der Direktzahlungen von der Produktionsmengen, sind die Produzenten in der Lage, die Produktionsmenge zu reduzieren.
Da die Direktzahlungen nicht gesenkt wurden, war die Kompensieurng höher als die tatsächlichen Verluste.
Dies führt dazu, dass die Exportmenge weiter reduziert wurde.
Dadurch steigt der Weltmakrtpreis weiter und die positiven Wohlfahrtseffekte sind sehr deutlich.
Angaben jeweils im Vergleich zu \cref{subsubsec:mindest}


\chapter{Umweltpolitik}
\setcounter{section}{0}
\section{Grundlagen der Umweltökonomie und Umweltpolitik}
Wenn Märkte vernünftig funktionieren, kommt eine optimale Ressourcenverteilung zu stande.
Allerdings werden per se keine Kosten für eine Verschlechterung des Allgemeinwohls berücksichtigt, somit kommt es teilweise zu sehr starken Schädigungen des Allgemeinwohls.
Die Gesellschaft, bzw einzelne Dritte, werden für ihre Nachteile nicht kompensiert.
In solchen Fällen spricht man von Marktversagen, was ein Eingreifen des Staates erforderlich macht.

\subsection{Externe Effekte der Agrarproduktion}
In der Landwirtschaft hat man viele solche Effekte im Bereich der Umwelt (Biodiversität, Nitratbelastung).
Andere externe Effekte sind zb Abgase von Kraftfahrzeugen.
Es existieren auch positive externe Effekte, zB. bei Impfungen.
Dabei profitieren andere davon, da diese sich nicht so leicht bei uns anstecken können, bzw. weniger schwere Verläufe die Krankenkassen entlastet werden.

Die \ac{VWL} Kostenkurve hat somit eine anderen Verlauf, als die private Kosenkurve der Produzenten.
In der \ac{VWL} Kostenkurve sind neben den Kosten der Produzenten auch die Kosten der externen effekte enthalten.
Für ein Unternehmen ist es sinnvoll, Kosten zu externalisieren und Gewinne selbst abzuschöpfen.

Die Gesellschaft wird die Landwirtschaft immer weiter darauf drängen, negative externe Effekte zu reduzieren sowie weitere positive externe Effekte zu steigern.
Dies sieht man an den steigenden \ac{CC}-Auflagen.
Es gibt zB Diskussionen über die Methanemissionen bei Milchkühen und wie man diese reduzieren kann, da der Landwirtschaftliche Sektor bei weitem der Größte Emitent von Methan in Deutschland ist.

Aus diesem Grund werden viele externe Effekte von der Politik internalisiert.
Dies geschieht, zB über Steuern oder über Verteuerungen der Produktionskosten.
Dadurch soll das Produktionsniveau gesenkt werden, um negative Effekte für die Allgemeinheit zu reduzieren.
Dabei geht es teilweise auch darum, die Schäden der negativen externen Effekte über Zahlungen auszugleichen.

Bei positiven externen Effekte ist das privatwirtschaftliches Optimum der Menge deutlich geringer als die volkswirtschaftliche.
Dies besteht zB bei dem Thema Bildung.
Unter privatwirtschaftlichen Gesichtspunkten wäre es zB sinnvoll, in jungen Jahren bereits zu Arbeiten um Geld zu verdienen.
Die Bildung der jungen Arbeitskräfte verzögert somit den Markteintritt, sorgt aber für höhere Steuern mittelfristig (mehr Konsum)

Es gibt verschiedene Prinzipien der Umweltpolitik.

\subsection{Prinzipien der Umweltpolitik}
\begin{itemize}
	\item Verursacherprinzip
	\item Gemeinlastprinzip
	\item Vorsorgerprinzip
\end{itemize}

\subsubsection{Verursacherprinzip}
Der Verursacher soll auch für die Kosten der Umweltbelastung tragen.
Ein Landwirt, welcher zu viel Wirtschaftsdünger ausbringt, ist ein direkter Verursacher der Nitratbelastung des Grundwasserkörpers.
Die Konkurrenzsituation der Landwirte, welche diese dazu zwingt, möglichst kosteneffizient zu arbeiten, ist über die Nachfrageseite ein indirekter Verursacher der Nitratbelastung des Grundwasserkörpers.
Häufig wird versucht, den direkten Verursacher ausfindig zu machen und direkt von ihm tragen zu lassen.

Probleme entstehen, da zB CO\textsubscript{2}-Emissionen über Ländergrenzen hinweg Schäden anrichten aber für einen einzelnen Staat nicht greifbar sind.
%Dies ist bei indirekten Verursachern sehr schwierig, da z.B. die Belastung des Grundwassers mit Nitrat nicht einzelnen Verursachern zuzuordnen - auch wenn ein 

\subsubsection{Gemeinlastprinzip}
Die Kosten von Umweltbelastungen werden der Allgemeinheit, also dem Staat/Steuerzahler aufgebürdet.
Die Verursacher müssen somit die Schäden ihres Handelns also nicht direkt tragen.
Dies wird in der Regel verwendet, wenn das Verursacherprinzip nicht angewendet werden kann.

\subsubsection{Vorsorgeprinzip}
Umweltschäden, bzw. belastungen sollen gar nicht erst entstehen.
Wird häufig auch als Nachhaltigkeitsprinzip bezeichnet, was aber etwas zu kurz kommt.

Der Begriff Nachhaltigkeit kommt von der Forstwirtschaft, wobei die Grundidee ist, von den Zinsen und nicht vom Kapital zu leben.
Die Wirtschaft hat die Grenzen des Wachstums aufgrund der vorhanden Natur - und die Politik ist damit beschäftigt, diese zu schützen.
Die aktuelle Defintion des Begriffs Nachhaltigkeit kommt von 1987 (Brundtlandkommission)

\textit{Entwicklung zukunftsfähig zu machen, heißt, dass die gegenwärtige Generation ihre Bedürfnisse befriedigt, ohne die Fähigkeit der zukünftigen Generation zu gefährden, ihre eigenen Bedürfnisse befriedigen zu können.}


\subsubsection{Nachhaltigkeit}
\label{nachhaltig}
\begin{itemize}
	\item Ökologische Nachhaltigkeit
		
		Schonender Umgang mit natürlichen ressourcen, wie zB erhalt der Artenvielfalt und Klimaschutz.

	\item Ökonomische Nachhaltigkeit

		Man ist langfristig und nachhaltig in der Lage Wohlstand aus einem Unternehmen zueinander stehen.
		Teilweise stehen die verschiedenen Nachhaltigkeiten gegeneinander in Konkurrenz stehen.

	\item Soziale Nachhaltigkeit

		Die Mitgleider einer Gesellschaft werden an der Entwicklung der Gesellschaft beteiligt.
		Somit soll eine lebenswerte Gesellschaft erhalten werden/geschaffen werden.
		
\end{itemize}

Diese Begriffe sind natürlich relativ gut dehnbar


\subsubsection{Diskussion: Molkerei- und Fleischereiprodukte gelten als Klimakiller}
Ein durchschnittlicher Haushalt konsumiert im Jahr etwa 290 kg/Jahr Molkereiprodukte.
Aufgrund der Methanemissionen einer Milchkuh entstehen relativ viele  CO\textsubscript{2}-Äquivalente (1t CO\textsubscript{2})
Der Verbrauch von Fleisch ist bei etwa 110 kg/Jahr, was etwas mehr als 450 kg CO\textsubscript{2}.

\todo{hier ist evtl das ein oder andere in Agrarpolitik zu verschieben}
\section{umweltpoltische Instrumente}

\subsection{Umweltabgaben und Umweltsteuern}



\subsection{Umweltauflagen}
Umweltauflagen sind Ordnugnsrechtliche Auflagen.
Es werden keine marktwirtschaftlichen Instrumente verwendet, somit besteht kaum bzw kein Spielraum.
Es ist zB nicht erlaubt Heizöl mit höherem Schwefelgehalt zu verbrennen, daher sind diese Regelungen die am stärksten einengende Form der Umweltpolitik.
Aktuell basieren rund 90 \% der Umweltpolitik auf Ordnungsrecht bzw Umweltauflagen.

Dies wurde gemacht, da über solche Vorgaben schnelle Erfolge erzielt werden können.

Zukünftig werden immer mehr Auflagen marktbasierte Instrumente nutzen.

Erstmal wird jeder Emittent wird gleich behandelt.
Dies hört sich ersmtal plausibel an, allerdings ist so ein Vorgehen durch Ineffizienz (und entsprechende Wettbewerbsverzerrung) geprägt.
Bei Umweltbedrohung, welche ein schnelles handeln erfordern, ist dies vorgehen praktisch alternativlos.

Die Kostenineffizenz kommt daher zu Stande, dass bei einem Verursacher die Kosten zur Reduktion sehr sehr hoch sind und bei einem anderen deutlich geringer (Bsp Emissionen).
Wenn die Auflagen eine Halbierung vorsehen, müssten beide Betriebe ihre Emissionen um die Hälfte reduzieren.
Dürten die Betriebe die Emissionsrechte handeln, könnte der Betrieb, der deutlich einfacher Emissionen verringern kann, diese Rechte an den anderen Betrieb verkaufen.
Dadurch können aus \ac{VWL}-Sicht sehr viele Kosten gespart werden.

Betriebe welche kurz vor eine Investition in dem Bereich stehen, werden bevorzugt, Betriebe welche gerade investiert haben, werden stark benachteiligt.
Des weiteren ist für manche Betriebe eine bestimmte Technik nicht Stand der Technik, bzw. wirtschaftliche sinnvoll, für andere Betriebe aber schon.
Wenn diese technik vorgeschrieben wird, wird der Betrieb, für den es sich nicht wirtschaftlich sinnvoll ist, deutlich benachteiligt.

Daher sind Umweltauflagen mittel- und langfristig nachteilhaft für den technischen Fortschritt.
Dabei spricht man dann von dynamischer Kostenineffizienz.

Als Verursacher aus Sicht der Umweltauflagen gilt der technische Emittent.
Dies greift aber zu kurz, da auch die vor- und nachgelagerten Aktivitäten in umweltgerechte Bahnen gelenkt werden müsste.
Solche Aspekte werden im Bereich der Umweltpolitik sehr wichtig werden.
Daher sollte der Blick auf die Wertschöpfungskette gerichtet sein und nicht nur auf den Emittenten, um auch in Zukunft einen guten Umweltschutz machen zu können.

Bei absolutem Verbote spielen Kostenunterschiede keine Rolle, da durch unterschiedliche Kosten keine Verlagerung stattfinden kann.
Die Verbesserung der Umweltsituation mittels Umweltauflagen kann sehr teuer werden, da es zu hohen Kostenineffizenzen kommen kann.

\section{Ökobilanz von Lebensmitteln}

Die \ac{UN}-Nachltigkeitszielsetzung ist die grundlage für die der \ac{EU}.
Die Aspekte sind die gleichen wie bei Nachhaltigkeit, siehe \cref{nachhaltig}.
Die Herausforderung dabei ist, dass die ökologische Aspekte der Nachhaltigkeit messbar zu machen.
Bei manchen Aspekten ist dies relativ einfach, bei anderen deutlich aufwändiger.
Dabei geht es nicht nur um die technische Hürden, sondern auch um die Probleme der Verfälschung der Ergebnisse (so beeinflusst ein Tier/Mensch über die Atmung die CO\textsubscript{2}-Konzentration in ihrer Nähe).
Weltweit trägt die Landnutzung etwa 23 \% der Klimagasemissionen, in Deutschland etwa 12 \%.

\ac{LCA} werden benötigt, um eine Ökobilanz zu erstellen.
Da bei solchen Betrachtung werden Ver- und Entsorgungsprozesse, Transporte, Energieerzegung und ähnliches betrachtet.
Eine Externaisierung soll als Lösung ausgeschlossen werden.
Dies dient dann als Grundlage, um Prozessoptimierungen vorzunehmen und sind die Grundlage von \textit{Blauen Engel} und anderen Verpackungsgesetzten.

Umweltbezogene Daten sind häufig nicht öffentlich und/oder nur schwer zu recherchieren bzw nur über eine Messung zu ermitteln.
Bei dem \ac{UBA} gibt es die meisten öffentlich verfügbaren Daten.

In Süddeutschland wird CO\textsubscript{2}-neutrale Milch angeboten.
Dies ist nur über Zertifikatshandel darstellbar, da die Milchproduktion derzeit technisch nicht CO\textsubscript{2}-neutral gestaltet werden kann.

 \subsection{Vertragsnaturschutz}
 \todo{Kapitel 3.1}
 Seit den 70er Jahren geht die Artenvielfalt deutlich zurück, nachdem diese anfang des 20. Jahrhunderts stark gestiegen ist (invasion fremder Arten?)
 Seit 2004 ist der Bestand der Feldlerche um etwa 50 \% zurück gegangen.
 Anhand solcher Indikatoren wird die politik überprüft und entsprechende agrarpolitischen Maßnahmen getroffen.
 Der Rückgang der Feldvogelarten ist in allen \ac{EU}-Ländern vorhanden und korreliert mit der intensivität des Anbaus.

 Seit den 70er ist ein stärker Rückgang der Vögel im Ackerland zu beobachten.
 Auch wenn dieser Rückgang zuletzt etwas langsamer war, ist der Trend nicht gestoppt bzw umgekehrt, wie das politische ziel ist.

Grenzertragsstandorte sollen aus der Nutzung heraus fallen, da diese für den Naturschutz vorteilhaft sind und für die Landwirtschaft der Verlust vertretbar ist.
Durch die intensive Landwirtschaft hat sich der Lebensraum für viele Tiere deutlich geändert.

Der Vertragsnaturschutz versucht die Betriebe dazu zu bringen, Flächen so zu bewirtschaften, wie es früher üblich war.
Dadurch soll der Artenschutz gefördert werden.

Die meisten Vertragsnaturschutzprogramme werden von der \ac{EU} (mit)bezahlt.
Häufig ist eine kofinanzierung mit dem Land nötig, in wenigen Fällen werden Programme auch komplett vom Land finanziert.

Die teilnahme ist jeweils freiwillig.
Die Ausgleichszahlungen sind deutschlandweit einheitlich, daher sind die bestimmte Regionen für solche Programme nur sehr schwer wirtschaftlich.
Bei manchen Programmen sind diese auf bestimmte Regionen abgestimmt und nur in diesen verfügbar.

Die Anträge haben eine Laufzeit von 5 Jahren.
Die Pachtflächen können nur unter \ac{VNS} genommen, wenn der Pachtvertrag entsprchend lange läuft.
Diese Laufzeit ist gewünscht, da viele Maßnahmen eine gewisse Zeit benötigt wird, um einen guten Effekt zu erzielen.
Nach Ablauf der Midestvertragslaufzeit wird der Vertrag nur jährlich verwendet.

Die Bewirtschaftung ist in der Zeit des \ac{VNS} mit strikten Regeln eingeschränkt, auch wenn die Ziele teilweise anders eingehalten werden könnten, bzw ein genaues Stichdatum sehr nachteilig sein kann.
Die meisten \ac{VNS} sind auf Einzelflächen ausgelegt, es gibt aber auch Programme, welche auf die komplette Betriebsfläche, bzw 90 \%.
Häufig sind da dann auch unterschiedliche Vorgaben innehralb des Betriebes möglich.

\ac{VNS} ist zusätzlich zu den Direktzahlungen (1. Säule).
Auf einer Fläche können aber verschiedene \ac{VNS}-Maßnahmen generell nicht kombiniert wird.
Biobetriebe bekommen häufig nur eine verringerte Zahlung aus dem \ac{VNS}.

Der \ac{VNS} ist in SH etwa doppelt so viel Grünlandfläche als Ackerland vorhanden.
Es wird auch die Weidehaltung gefördert, da dies für die Artenvielfalt vorteilhaft ist und bestimmte Biotope nicht mit einem Mahd möglich ist.

Je geringer der Ertrag ist, desto geringer ist in der Regel die Artenzahl
Da die Artenzahl gesteigert werden soll, muss der Ertrag reudziert werden.
Ertrag und Qualitätseinbußen müssen von \ac{VNS} bezahlt werden - reicht aber nur in bestimmten Fällen wirtschaftlich.

Die meisten Wiesenvögel bauen ihre Neste im März und April auf.
Die Kükenaufzucht ist bis Anfang/Mitte Juli abgeschlossen.
Im \ac{VNS} sind die Sperrfristen für bestimmte Tätigkeiten auf diesem Grund eingeschränkt (Grünlandpflege nur bis zum 1.4, Mahd erst nach dem 21.6).

Brachen in der Landwirtschaft waren früher normal (10 bis 18. Jahrhundert), heute werden diese über die Politik gefördert.
Es gibt aber heutzutage auch eher hybride Lösungen, also zB Waldrandstreifen.

Durch Ackerbrachen (Blühflächen) wird idR die Artenvielfalt gefördert, da diese auch häufig Nahrungsangebote darstellen.
Wildpflanzenmischungen sind sehr sehr teuer, und nur schwer verfügbar, daher häufig nur überregional verfügbare Mischungen, welche preiswert zu bekommen sind.
Diese Mischungen sind deutlich besser als eine Selbstbegrünung, können auch von der Behörde einfach besser kontrolliert werden als eine Selbstbegrünung.
Im \ac{VNS} besteht ein Durchfahrverbot, ist es häufig nicht sinnvoll, kleine Biotope zu erstellen, da ein neues Vorgwende erstellt werden müsste.


\section{Gewässerschutzpolitik}
\todo{Kapitel 4}

Wasser in guter Qualität ist bei uns selbstverständlich, in anderen Regionen nicht.
Qualität muss erhalten bleiben, daher viele Auflagen und Vorschriften.
So gibt es zB die Nitratrichtlinie 19991 zum Schutz von Gewässer vor Nitrat aus landwirtschaftlichen Quellen.

Mit der Wasserrahmenrichtlinie soll der Grundwasserzustand sich nicht verschlechtern dürfen.
Weil dies nicht erreicht wird, müssen die Vorgaben weiter verschärft werden.

Im östlichen Hügelland und in der Marsch ist das Grundwasser über die schweren Böden gut geschützt.
Dies trifft nicht auf die Geest mit den leichten, sandigen Böden zu.

Aufgrund der Nähe zu den Meeren führt in den tiefen der Grundwasserkörper Salzwasser und nicht Süßwasser.
Die oberflächen Wasser (Flüsse, Seen, küstennahe Meer) sind in einem schlechten Zustand, der Grundwasserkörper auch.

Die größten Niederschläge fallen häufig im August, der April ist der trockenste Monat.
Diese Daten werden aus dem langjährigem Mittel (30 Jahre) ermittelt.
Dabei geht es aber nicht um den gleitenden Mittel, sondern um feste Abstände (19991-2020)
Der November ist in der Wasserrechnung der erste Monat des Jahres, da die Boden in einem ähnlichen Zustand befinden, da die Vegetation abgeschlossen ist.
Somit ist der Grundwasserstand am niedrigsten Stand des Jahresverlauf.
Das trockene Jahr 2018 ist bisher (2022) weiterhin nicht ausgeglichen.
Die Tiefenverlagerung auf schweren Böden ist bei 4 dm/a, bei leichten Böden bei 17 dm/a.

Bei den Meßstellen wird der Grundwasserkörper gemessen.
Dabei werden die Quellen von Stoffen aus einem großen Radius gemessen, insbesondere je tiefer die Messung durchgeführt wird.
Nitratauswaschungen sind im Wald am geringste, dann kommt das Grünland und am \glqq schlimmste\grqq{} ist der Acker.

\subsection{Stickstoffeintrag durch Sickerwasser}
Die Stickstoffeintragung in den Grundwasserkörper ist in Deutschland lokal sehr unterschiedlich.
Schleswig-Holstein ist im Vergleich zu Deutschland eine höherer Verschmutzer.
Es ist in den letzten Jahren keine Verbesserung zu erkennen.

Intensive Viehhaltung und Gemüseanbau sind häufige Verursacher von hohen Stickstoffbelastung.
In trockenen Regionen sind die geringen Sickerwasserraten problematisch, da das Nitrat nicht verdünnt wird.
So ist trotz geringen Stickstoffüberschüssen der Grundwasserkörper in einem schlechten Zustand und kaum zu verbessern.

\subsection{Messtechnik}
Bei den Messstelle wird wasser abgepumpt und temperatur und einige andere Parameter beobachtet.
Sobald die Werte konstant sind, werden die Proben gezogen, dies dauert in der Regel zwischen 30 und 60 Minuten.
Manshce Messtellen werden im Frühjahr und andere im Herbst beprobt.

\subsection{Grundwassermonitoring}
Es gibt ein Pool von Messtellen und davon soll jeweils ein Teil für eine bestimmte Aufgabe verwendet werden.
Es gibt kein eigenes Nitratmessnetz, sodass die Messstellen von anderen Messnetzen genommen wird.
Dies ist nicht optimal, da man so kompromisse eingehen muss.

\subsection{rote gebiete}
In den ergebnissen ist zu erkennen, dass in Schleswig Holstein der Grundwasserkörper unter der Geest einen schlechten Zustand hat, unter dem östlichen Hügelland und Marsch aber nicht.
Mit den roten Gebieten werden vor Gericht noch einige Rechtsstreite ausgetragen.
im Vergleich zu den Grundwasserkörper sind die roten Düngergebiete sehr klein.
Daher ist eine Ausweitung der roten Gebiete, wie in einigen prozessen geschehen, nicht unwahrscheinlich.

In den Messstellen in Schleswig Holstien werden an den Nitratmessstellen leicht zurück gehende Nitratwerte gemessen.
Allerdings sind die Werte immer noch zu hoch und die Werte müssten schneller fallen.
Grünland ist dabei deutlich besser als Ackerbau zu bewerten.

Je früher wirtschaftsdünger ausgebracht wird, desto besser wird dieser ausgenutzt, optimal Februar bis April, je nach Frucht, Mais und Grünland ist Mai nur minimal schlechter.
In Dänemarkt muss der Landwirt bei dem Landhändler seine Stickstoffbilanz vorzeigen, um Stickstoffdünger zu kaufen.
Eine Beziehung zwischen hoftorbilanz und hoher Viehbesatz ist mit R=0,24*** als relativ gering zu bewerten.
Es gibt somit andere, deutlich wichtigere Quellen.
Der Mineraldünger hat einen Einfluss von R=0,42.
Daher ist der Ansatz von Dänemarkt, den Verkauf von Mineraldünger strenger zu überwachen, besser geeignet, als die Reduzierung der Viehbesatzdichte.


Wir importieren und exportieren sehr viele Lebensmittel, daher ist eine Reduzierung der lokalen Produktion als negativ anzusehen, da in anderen Regionen die Lebensmittel zu schlechteren Bedingungen produziert.
Die moderne technik wurde bereites vor 50 Jahren als sinnvoll erachtet, es könnte also schon lange Flächendeckend eingesetzt werden.
\todo{Mirschrift nicht mehr möglich, da Folien nicht angezeit werden}

%\begin{itemize}
%\end{itemize}
\end{document}

























