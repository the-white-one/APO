\documentclass[11pt]{article}
\usepackage[utf8]{inputenc}
\usepackage[ngerman]{babel}
\usepackage[a4paper, left=2cm, right=5cm, top=2cm]{geometry}

\title{Mitschrift APO WiSe 21/22}
\author{Tibor Weiß}
\begin{document}
\maketitle
\section{Einfürhung in die Agrarpolitik}

\subsection{Grunddaten zur deutschen Land\-wirt\-schaft und Wert\-schöpf\-ung}
%todo Mitschrift kommt noch, mit nächstem Punkt in der VL angefangen.


\subsection{Landwirtschaftliche Strukturwandel und Determinanten}

\subsubsection{Was ist strukturwandel?}
\begin{itemize}
	\item{Änderung von Daten im Sektor, zB Betriebsgröße, Strukturen, Produktivitäten\ldots}
	\item{Bewertung von Strukturwandel (VW-Sicht)}

		Generell ist Strukturwandel erwünscht, da so eine produktivere Wirtschaft ermöglicht wird. Dies führt dazu, dass \glqq schwache\grqq{}  Betriebe ausscheiden, da diese nicht in der Lage sind, ein nachhaltiges Einkommen zu erzielen.
Landwirtschaft ist der Strukturwandel sehr langsam, aufgrund der langen Investitionszyklen
	\item{Soziale Härten durch Anpassungen}

Immer weniger Betriebe im Velraufe der zeit (beobachtung)

\end{itemize}


\subsubsection{Determinanten}

Druck auf die Produzenten
\begin{itemize}
	\item{Spezialisierung und Skaleneffekte}

		Skaleneffekte nutzen, durchschnittliche Stückkosten senken (Spezialisierung bzw Vergrößerung der Betriebe)
	\item{technischer Fortschritt (auch biologisch, chemisch und organisatorischer)}

		Möglichkeit der Produkivitätssteigerung durch Mechanisierung (weniger Handarbeit und schnellere Arbeitserledigung)
		Dadurch erhöhen sich die Produktionsmengen und die Preise fallen. Durch eine gesteigerte Nachfrage, kann der Effekt der fallenden Preise ausgeglichen werden, bzw verringert werden.
		In der Landwirtschaftlichen Primärproduktion ist eine Steigerung der Nachfrage eher gering.
	\item{Außerlandwirtschaftliche Beschäftigungsmöglichkeiten}

		Abwanderung von Arbeitskräften, Notwendigkeit der Produktivitätssteigerung
	\item{ungesicherte Hofnachfolge}

		Familienbetriebe werden bei fehlendem nachfolger aufgegegben. Bei nicht inhaber geführten Betrieben (Genossenschaften, AGs\ldots) wird ein neuer Verwalter oä eingestellt
	\item{internationaler Wettbewerb}

		Produkte werden auf dem Weltmarkt gehandelt und beeinflussen daher die lokalen Preise - generell Preise eher nach unten

		Aktuell (Herbst 2021) liegen die Weltmarktpreis deutlich höher.
		Wenn der Binnenmarkt am Welthandel teilnimmt, wird der Preis des Welthandels diktiert.
		Es entsteht eine Konsumentenrente (idR) oder Produzentenrente (aktuell) und führt zu Importen bzw. Exporten.

	\item{gesetzliche Auflagen}

	\item{Gesellschaftliche Anforderungen}

		82 Millionen Agrar-Experten in DE
	\item{kritische öff. Diskussion über die Landwirtschaft}
\end{itemize}



\end{document}
